\section{flash读写及分区}
\subsection{flash}
usage: flash <子命令> [可选参数 <值>]

\begin{table}[H]
%\centering
\setlength{\parindent}{0pt}
\begin{tabular}{|c|l|} \hline
子命令名称 & 命令说明 \\ \hline
%select & 选中某个flash设备,作为其他flash命令的操作对象 \\ \hline
dump & 查看flash页内容,以page为单位,包括oob \\ \hline
read & 从Flash中加载数据到DDR \\ \hline
write & Flash中写数据from DDR \\ \hline
erase & 以block为单位,擦除flash一块内容 \\ \hline
scanbb & 扫描指定分区上的坏快 \\ \hline
\end{tabular}
%\caption{Flash子命令}
\end{table}

\begin{table}[H]
\setlength{\parindent}{0pt}
\begin{tabular}{|c|l|} \hline
参数 & 功能 \\ \hline
-a & 设定起始地址\\ \hline
-l & 设定长度\\ \hline
-p & 设定分区,不与-a和-l并用\\ \hline
-m & mem的起始址址\\ \hline
-f & 强制erase(无论是否存在坏块)\\ \hline
-c & 擦除的同时写入cleanmark\\ \hline
\end{tabular}
\end{table}

命令使用示例:\\
\begin{itemize}
\item flash erase -a 1M -l 32K
\item flash erase -a 100block -l 16block -f
\end{itemize}

\subsection{part}
usage: part [options]

参数说明:

\begin{table}[H]
\begin{tabular}{|c|l|}
\hline
参数 & 功能 \\ \hline
-l & 显示分区表 \\ \hline
%-i [N] & 显示分区N的具体信息 \\ \hline
\end{tabular}
\end{table}

\subsection{ls}
显示当前分区的具体信息

\subsection{cd}
切换分区

\section{MMC/SD卡操作}
\subsection{mmc}
usage: mmc $<$子命令$>$ [options]

\begin{table}[H]
%\centering
\setlength{\parindent}{0pt}
\begin{tabular}{|c|l|} \hline
子命令名称 & 命令说明 \\ \hline
scan & 扫描所有MMC/SD设备,解析并打印设备信息 \\ \hline
%select & 选中某个MMC/SD设备,作为其他MMC/SD命令的操作对象 \\ \hline
dump & 查看MMC/SD上的数据,每次显示一个block \\ \hline
%read & 从MMC/SD中加载数据到DDR \\ \hline
%write & MMC/SD中写数据from DDR \\ \hline
%erase & 以block为单位,擦除MMC/SD一块内容 \\ \hline
\end{tabular}
%\caption{MMC子命令}
\end{table}

MMC/SD命令选项及参数介绍:
\begin{table}[H]
\setlength{\parindent}{0pt}
\begin{tabular}{|c|l|} \hline
参数 & 功能 \\ \hline
-a & 设定起始地址\\ \hline
%-l & 设定长度\\ \hline
%-m & mem的起始址址\\ \hline
\end{tabular}
\end{table}

\section{网络连接}

\subsection{ifconfig}
usage: ifconfig [interface] [address] [netmask <address>] [hw [HW] <address>]

选项及参数介绍:
\begin{table}[H]
\setlength{\parindent}{0pt}
\begin{tabular}{|l|l|} \hline
选项 & 功能描述 \\ \hline
interface & 指定网络设备对象,如``eth0''。缺省为系统中第一个网络设备。 \\ \hline
address & 配置网络设备IP地址为address \\ \hline
netmask <address> & 配置netmask为address \\ \hline
hw [HW] <address> & 配置设备的MAC地址为address。其中HW缺省为``ether'' \\ \hline
\end{tabular}
\end{table}

不加任何option时显示interface的信息,具体包括:
\begin{enumerate} \setlength{\itemsep}{-\itemsep}
\item NIC芯片名称(ID及字符串表示)
\item PHY信息:ID、地址
\item 连接状态,包括速率
\item RX/TX bytes
\item error count
\end{enumerate}

\subsection{ping}
usage:ping [DestIp] 

若DestIp不指定,则默认的server ip

\subsection{tftp}
usage:tftp [options] [filename]

参数介绍:
\begin{table}[H]
\setlength{\parindent}{0pt}
\begin{tabular}{|c|l|} \hline
选项 & 功能描述 \\ \hline
 -s & 设定服务端IP \\ \hline
 -m & 下载的内容放在内存里,即,不烧录到flash上 \\ \hline
\end{tabular}
\end{table}

\subsection{dhclient}
usage:dhclient [options]

参数介绍:
\begin{table}[H]
\setlength{\parindent}{0pt}
\begin{tabular}{|c|l|} \hline
选项 & 功能描述 \\ \hline
 -s & 同时将Server IP更新为DHCP Server \\ \hline
\end{tabular}
\end{table}

\section{串口协议及工具}

\subsection{kermit}
usage:kermit [options]

作用概述:串口文件传输%\footnote{目前kermit和ymodem仅支持下载}

选项及参数介绍:
\begin{table}[H]
\setlength{\parindent}{0pt}
\begin{tabular}{|c|m{10cm}|} \hline
选项 & 功能描述 \\ \hline
 -m [address] & 将下载的数据放在内存中,而不直写到storage(如Flash)上。其中address为可选参数,表示memory地址;若不指定address,则由系统自动分配一块空间。 \\ \hline
\end{tabular}
\end{table}

\subsection{ymodem}
usage: ymodem [options]

作用概述:串口文件传输%\footnote{目前kermit和ymodem仅支持下载}

选项及参数介绍:
\begin{table}[H]
\setlength{\parindent}{0pt}
\begin{tabular}{|c|m{10cm}|} \hline
选项 & 功能描述 \\ \hline
 -m [address] & 将下载的数据放在内存中,而不直写到storage(如Flash)上。其中address为可选参数,表示memory地址;若不指定address,则由系统自动分配一块空间。 \\ \hline
\end{tabular}
\end{table}

\section{Graphics和Display}
\subsection{lcd}
usage:lcd [options]

参数介绍:
\begin{table}[H]
\setlength{\parindent}{0pt}
\begin{tabular}{|c|l|} \hline
选项 & 功能描述 \\ \hline
-l [all] & 列出LCD的video mode。all表示所有video mode,不加则仅显示当前的video mode \\ \hline
-s <N> & 将当前LCD的video mode设置为第N种mode。 \\ \hline
\end{tabular}
\end{table}

\section{memory读写及指令跳转}

\subsection{mem}
\begin{table}[H]
%\centering
\begin{tabular}{|c|l|}
\hline
命令名称 & 命令说明\\ \hline
read & 显示memory数据\\ \hline
write & 将数据写入memory\\ \hline
set & 将某个memory空间写入值\\ \hline
\end{tabular}
%\caption{md}
\end{table}

\subsection{go}
usage: go <address>

address跳转的目标地址,可十进制表示,也可十六进制表示。

示例: go 0xc000000 跳转到0xc000000处执行。

\section{系统配置}
\subsection{sysconf}
\begin{table}[H]
%\centering
\begin{tabular}{|c|l|}
\hline
命令名称 & 命令说明\\ \hline
-r <all|net|boot> & sysconf reset \\ \hline
\end{tabular}
%\caption{md}
\end{table}

\section{其他命令}
\subsection{help}
列出g-bios系统中当前所有可用的命令

\subsection{led}
LED灯测试
