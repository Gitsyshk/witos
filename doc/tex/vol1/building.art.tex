\section{g-bios配置}
与编译Linux kernel类似,进入在g-bios源码目录,执行make PLAT\_defconfig(PLAT指的是具体硬件平台名称,例如s3c6410\_defconfig或者beagle\_defconfig, g-bios所支持的各硬件平台的默认配置文件位于g-bios源码build/configs目录下),用默认的选项编译g-bios,然后执行make进行编译,如需要将编译产生的image文件拷贝到tftpboot目录下,还需要执行make install命令。如果需要修改默认的编译选项,可以直接执行make menuconfig,在随后出现的GUI中进行配置。目前的2.5版本暂不支持GUI配置方式。

\subsection{基于命令行的配置方式}
切换到g-bios源码目录下,然后执行如下命令:
\begin{lstlisting}[language=bash, numbers=none]
$ make ${BOARDNAME}_defconfig
\end{lstlisting}
其中BOARDNAME为某个具体的硬件平台名称,如beagle、mw61、mini2440等。

\subsection{基于图形界面的配置方式}
切换到g-bios源码目录下,然后执行如下命令:
\begin{lstlisting}[language=bash, numbers=none]
$ make menuconfig
\end{lstlisting}

\subsection{配置选项详解}
接下来分析一个各个配置选项的功能作用。

Platform:	g-bios运行的platform,可以是at91sam9263、at91sam9261、s3c2410、s3c2440或s3c6410等,这是目前g-bios支持的几个Platform。Toolchain:编译g-bios源码所选用的编译工具,默认使用的是lablin源码包编译生成的toolchain,也可以手工修改为系统上已有的toolchain(注:Toolchain要支持EABI)。Image Patch编译g-bios后生成的image路径,默认为/var/lib/tftpboot目录。Server IP服务器IP,local IP开发板IP,将二者设为同一网段。此二项,也可不配。MAC Addr此项不用理会,Nfs Path:g-bios引导内核时,如用nfs加载rootfs时,指定rootfs路径,默认路径$\sim$/maxwit/rootfs。Flash ECC mode选择ECC校验模式(硬件ECC,软件ECC,也可不使用ECC)。IRQ/Polling Mode g-bios使用中断模式还是非中断模式(Polling)。

g-bios配置程序所完成的功能:
\begin{table}[htbp]
\centering
\setlength{\parindent}{0pt}
\begin{tabular}{|c|l|l|}
\hline
类别 & \multicolumn{1}{|c|}{选项} & \multicolumn{1}{|c|}{功能说明} \\ \hline
\multirow{3}{*}{general} & Platform & g-bios运行的目标Platform \\ \cline{2-3}
		& TooLchain & 编译g-bios的编译工具 \\ \cline{2-3}
		& Image Path & 编译生成image的目录 \\ \hline
\multirow{4}{*}{Network} & Server IP & 服务器IP \\ \cline{2-3}
		& Local IP & 目标机IP \\ \cline{2-3}
		& MAC addr & MAC地址\\ \cline{2-3}
		& NFS root path &  lablin的rootfs路径\\ \hline
\multirow{3}{*}{Flash ECC Mode}   & Hardward & 支持硬件ECC\\ \cline{2-3}
		& Software & 软件ECC\\ \cline{2-3}
		& None & 无ECC较验\\ \cline{2-3}
\multirow{2}{*}{IRQ/Polling Mode} & IRQ Enabled & 支持中断 \\ \cline{2-3}
		& Polling Mode & 查询模式(非中断)\\ \hline
\end{tabular}
\end{table}

%\begin{itemize}
%\item Gerneral
%\item Tophalf
%\item Uart
%\item Memory
%\item Flash
%\item NetWork
%\item Interrupt
%\item Logo
%\item Boot
%\end{itemize}

\section{编译}
%\subsection{``make'' and ``make install''}
上面通过configure配置的g-bios编译特性,生成了Makefile。本节将编译g-bios。
\begin{lstlisting}[language=bash,numbers=none]
$ make
$ make install
\end{lstlisting}
编译后会在/var/lib/tftpboot(configure中配置的Image Path)目录下生成g-bios-th.bin和g-bios-bh.bin二个文件。
