\section{g-bios简介}
%\section{g-bios: An Open Source Bootloader Project}
%MaxWit开放实验室(MaxWit Open Lab)是由多家公司资助成立的,致力于研发开源项目和探讨软件开发技术的公益性组织。2008年1月正式成立于上海浦东张江高科,目前开放实验室成员主要来源于Google、Intel、 TI、AMD、华为、Cisco、飞利浦等公司资深研发人员以及清华、浙大、上交大、中科院等科研院校的师生。

g-bios(以下简称g-bios)是由Intel、IBM、Qualcomm、AMD等公司的几名软件工程师与开源社区共同研发的一个Bootloader\footnote{或者说是一个嵌入式系统的BIOS,相当于PC机的BIOS+Bootloader。}。g-bios不但借鉴了几乎所有主流BSP/BIOS/Bootloader的优点,而且加入不少独创的特性,包括:
\begin{enumerate}\setlength{\itemsep}{-\itemsep}
\item 自动检测有待烧录的image文件类型,并智能自动烧录。
\item 支持多种文件系统,包括YAFFS2、JFFS2、CRAMFS、UBI、NFS等。
%\item 支持两种用户界面:GUI(类似传统PC BIOS)和命令行模式(面向嵌入式系统)。
\item 命令行自动补全(Tab)键及历史记录(上、下键)支持。
\item Flash(MTD)分区支持,帮助Linux、Android内核识别分区。
\item 自动设置Linux内核启动参数(Linux kernel command line),极大地降低了参数设置的复杂度并减少了启动出错的概率。当然,同时也支持手动设置,以满足特殊要求。
\item 常用命令具有记忆功能。如boot命令,它能记住用户输入的参数,以后只需简单输入boot即可。
\item 引入全新的架构及NB技术(即Never Burn-down,又称``烧不死''技术)。
%核心设计思想是:把g-bios分为上半部分和下半部分,上半部分以最小的代码量完成CPU和Memory的初始化,并实现引导下半部分的功能;下半部分为g-bios主体。上半部分设计简单,调试周期短,完成后就固化在特定的引导区中不再更改;
开发人员可在没有仿真器的情况下大胆开发
Bootloader
%下半部分代码(即g-bios%主体)
。事实上,只需一根串口数据线应能轻松完成整个g-bios的开发。启动代码的地址无关性带来的麻烦?没有了!因为bug或不小心改错了代码,甚至是数据线连接问题而导致启动黑屏?也不可能出现了!
%在调试完成并正试发布的产品时,若有必要,也可将上下两部分可合成一个整体——只需一个命令重新编译即可。
\item 支持完整的中断机制。开发者可简单地通过一个编译选项选择IRQ或Polling两种模式的中的任意一种。
\item 优秀的网络子系统,并提供符合POSIX规范的Socket API,方便二次开发。
%\item 优秀的软件架构及子系统设计,包括:中断、网络、Flash、USB子系统,等等。
%\item 集成类似PC机版本的Video BIOS。
%\item 支持基于龙芯的PC机及嵌入式系统。
\item 支持
%嵌入式系统几乎所有
多种常用外设,包包括:WDT、UART、NAND、NOR、SD/MMC、USB、LCD、Touchscreen,...
\item 集成硬件调试/测试程序,大大提高了bring-up的工作效率。
\item 完美支持Google Android操作系统,简化Android的系统移植过程。
\item 支持图形化配置,不但让新手很容易上手,而且使g-bios的移植和开发过程变得更简单。
\end{enumerate}

更多详情,请登录项目主页http://maxwit.googlecode.com或ChinaUnix论坛上的g-bios版块(http://bbs.chinaunix.net/forum-238-1.html)。

\section{获取g-bios源码}
请确认git(一个版本管理软件)已经安装,然后执行如下命令:
\begin{lstlisting}[language=bash,numbers=none]
$ git clone git://github.com/maxwit/g-bios.git
\end{lstlisting}
此时会在当前目录(方便描述起见,假定为HOME目录)下将会创建一个名为 ``g-bios''的目录,该目录中为g-bios源码。
%\section{如何参与g-bios开发}
%g-bios开源社区采用maillist和bbs相结合的方式,任何人都可以通过这两种方式把自己的代码递交给g-bios项目维护者。若对文档有任何疑问或改进也可联系我们。
%  \begin{table}[htbp]
%  \centering
%  \setlength{\parindent}{0pt}
%  \begin{tabular}{|c|c|}
%  \hline
%  g-bios论坛 &\small http://linux.chinaunix.net/bbs/forum-70-1.htm \\
%  \hline
%  g-bios邮件列表 &\small maxwit@googlegroups.com \\
%  \hline
%  g-bios项目维护者 &\smallConke Hu $<$conke.hu@gmail.com$>$ \\
%					 &\smallTiger Yu $<$tigerflying.yu@gmail.com$>$ \\
%					 &\smallFleya Hou $<$fleya.hou@gmail.com$>$ \\
%  \hline
%  文档编辑 &\small  \\
%  \hline
% \end{tabular}
% \end{table}

\section{g-bios体系结构}
% figure
