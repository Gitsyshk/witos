\section{stdio}
\begin{enumerate}
\item clear\_sreen
\begin{lstlisting}[language=c, numbers=none]
#include <stdio.h>

void clear_screen(void);
\end{lstlisting}

\item putchar
\begin{lstlisting}[language=c, numbers=none]
#include <stdio.h>

int putchar(int ch);

arguments:
	ch---a charater which need display.
return val:
	return this charater.
\end{lstlisting}

\item puts
\begin{lstlisting}[language=c, numbers=none]
#include <stdio.h>

int puts(const char *str);

description:
	dispaly a string and a newline.

arguments:
	str---a string which need display.
return val:
	if successed return 0, failed return -1.
\end{lstlisting}

\item gets
\begin{lstlisting}[language=c, numbers=none]
#include <stdio.h>

char *gets(char *str);

description:
	read a line form stdin into buffer pointer to by str, until a terminating newline.

arguments:
	str: a buffer's address.
return val:
	return str.
\end{lstlisting}

\item printf
\begin{lstlisting}[language=c, numbers=none]
#include <stdio.h>

int printf(const char *fmt, ...);

\end{lstlisting}

\item sprintf
\begin{lstlisting}[language=c, numbers=none]
#include <stdio.h>

int sprintf(char *buf, const char *fmt, ...);

\end{lstlisting}

\item snprintf
\begin{lstlisting}[language=c, numbers=none]
#include <stdio.h>

int snprintf(char *buf, int size, const char *fmt, ...);

\end{lstlisting}

\end{enumerate}

\section{IRQ}
\begin{enumerate}
\item irq\_register\_isr
\begin{lstlisting}[language=c, numbers=none]
#include <irq.h>

typedef int (*IRQ_DEV_HANDLER)(u32, void*);
int irq_register_isr(u32 irq, IRQ_DEV_HANDLER isr, void *dev);

description:
	register a isr into irq subsystem.

arguments:
	irq----irq number.
	isr----irq function.
	dev----device.

return val:
	reuturn 0, is successed. return -22, irq is too large, return -12, is no memory.
\end{lstlisting}

\item irq\_set\_trigger
\begin{lstlisting}[language=c, numbers=none]
#include <stdio.h>

int irq_set_trigger(u32 irq, u32, type);

description:
	set pin's/irq trigger mode.

arguments:
	irq---irq number.
	type---trigger mode
mode:
	IRQ_TYPE_NONE
	IRQ_TYPE_RISING
	IRQ_TYPE_FALLING
	IRQ_TYPE_BOTH -----falling or rising
	IRQ_TYPE_HIGH
	IRQ_TYPE_LOW

return val:
	if successed, return 0, otherwise return -1;
\end{lstlisting}

\item irq\_set\_handler
\begin{lstlisting}[language=c, numbers=none]
#include <irq.h>

description: set irq handle for each irq.

void irq_set_handler(u32 irq, IRQ_PIN_HANDLER irq_handle, int chain_flag);

arguments:
	irq----irq number.
	irq_handle---isr.
	chain_flag---irq pin is chained? if chained, chain_flag=1, otherwise chain_flag=0.
\end{lstlisting}

\item irq\_assoc\_intctl
\begin{lstlisting}[language=c, numbers=none]
#include <irq.h>

int irq_assoc_intctl(u32 irq, struct int_ctrl *intctrl);

description: set irq's int control.

arguments:
	irq---irq number.
	intctrl---irq's control.

return val:
	always return 0;
\end{lstlisting}

\item irq\_handle
\begin{lstlisting}[language=c, numbers=none]
#include <irq.h>

void irq_handle(u32 irq);

description: execute irq's isr.
\end{lstlisting}

\item irq\_handle\_level
\begin{lstlisting}[language=c, numbers=none]
#include <irq.h>

void irq_handle_level(struct int_pin *pin, u32 irq);

description:
	execute pin's irq_list.
arguments:

void irq_handle_edge(struct int_pin *pin, u32 irq);
void irq_handle_simple(struct int_pin *pin, u32 irq);
\end{lstlisting}
\item irq\_handle\_edge: 与~irq\_handle\_level~功能类型,仅触发方式不同。
\item irq\_handle\_simple: 与~irq\_handle\_level~功能类型,但不管触发方式。
\end{enumerate}

\section{String}
\begin{enumerate}
\item ~string2value~将字符串转化为数值。
	\begin{lstlisting}[language=c, numbers=none]
	#include <string.h>

	int string2value(const char *str, u32 *val);
	arguments:
		str---a string which need to convert.
		val---value after converted.
	RETURE VALUE:
		if convert successed, return length of string. otherwise reture -EINVAL
	\end{lstlisting}
\item dec\_str\_to\_val
	\begin{lstlisting}[language=c, numbers=none]
	#include <string.h>

	int dec_str_to_val(const char *str, u32 *val);
	arguments:
		str---a string which need to convert.
		val---value after converted.
	RETURE VALUE:
		if convert successed, return length of string. otherwise reture -EINVAL
	\end{lstlisting}

\item strchr
	\begin{lstlisting}[language=c, numbers=none]
	#include <string.h>

	char *strchr(const char *psrc, int c);
	arguments:
		psrc---a string
		c---a chartater which need find.
	RETURE VALUE:
		if successed, return the char's start address. otherwise reture NULL.
	\end{lstlisting}

\item strrchr
\item strcat
\item strncat
\item strcmp
\item strncmp
\item strlen
\item strnlen
\item memset
\item memmove
\item memcmp
\item memcpy
	\begin{lstlisting}[language=c, numbers=none]
#include <string.h>

long memcmp(const void *pdst, const void *psrc, size_t count);
	if pdest==psrc, return 0, otherwise return a value(+/-).
	\end{lstlisting}
\end{enumerate}

\section{network}
\begin{enumerate}
\item net\_set\_mac
\begin{lstlisting}[language=c, numbers=none]
#include <net.h>

int net_set_mac(struct net_device *ndev, const u8 mac_adr[])

description:
	set a network device's mac address.
arguments:
	ndev--network device.
	mac_adr--mac address's array.

return val:
	if successed, return 0; otherwise return -ENODEV
\end{lstlisting}
\end{enumerate}

\section{MTD}
\subsection{part}
\begin{enumerate}
\item part\_write
\begin{lstlisting}[language=c, numbers=none]
#include <flash/part.h>

long part_write()
\end{lstlisting}
\end{enumerate}

\subsection{Flash}
\begin{enumerate}
\item flash\_open
\begin{lstlisting}[language=c, numbers=none]
#include <flash/flash.h>

struct flash_chip *flash_pen(unsigned int num);
arguments:
	num---flash number
\end{lstlisting}
\item flash\_ioctl
\begin{lstlisting}[language=c, numbers=none]
#include <flash/flash.h>

int flash_ioctl(struct flash_chip *flash, int cmd, void *arg);

cmd:
	FLASH_IOCS_OOB_MODE
	FLASH_IOCG_OOB_MODE
	FLASH_IOCS_CALLBACK
	FLASH_IOC_SCANBB

arg:

\end{lstlisting}
\item flash\_close
\item flash\_write
\begin{lstlisting}[language=c, numbers=none]
#include <flash/flash.h>

long flash_write(struct flash_chip *flash, const void *buff, u32 count, u32 ppos);

u32 ppos: postion
\end{lstlisting}
\item flash\_read:读取数据,也可读取~OOB~
\begin{lstlisting}[language=c, numbers=none]
#include <flash/flash.h>

int flash_read(struct flash_chip *flash, void *buff, int start, int count);
arguments:
	flash: flash which read form data.
	buff: read data's write to memory's address.
	start: flash's address.
	count: request read's data size.
\end{lstlisting}
\item flash\_erase
\begin{lstlisting}[language=c, numbers=none]
#include <flash/flash.h>

int flash_erase(struct flash_chip *flash, u32 start, u32 size, u32 flag);
\end{lstlisting}
\end{enumerate}

\section{memory}
\begin{enumerate}
\item malloc
\begin{lstlisting}[language=c, numbers=none]
#include <malloc.h>

void *malloc(u32 size);

argument:
	u32 size: request memory size.

return: if alloce success, return virtual address.
\end{lstlisting}
\item free
\begin{lstlisting}[language=c, numbers=none]
#include <malloc.h>

int free(void *p);.
\end{lstlisting}
\item zalloc: request memory and clear 0.
\begin{lstlisting}[language=c, numbers=none]
#include <malloc.h>

void *zalloc(u32 len);
\end{lstlisting}
\item dma\_malloc, return pa and va, va=pa
\begin{lstlisting}[language=c, numbers=none]
#include <malloc.h>

void *dma_malloc(size_t len, u32 *pa);
\end{lstlisting}
\item gk\_init\_heap. initialize heap memory.
\begin{lstlisting}[language=c, numbers=none]
#include <malloc.h>

int gk_init_heap(u32 start, u32 end);

arguments:
	start: heap start address.
	end: heap end address.
returen val:
	if successed, return 0, otherwise return -EINVAL.
\end{lstlisting}
\end{enumerate}

\section{Misc}
\begin{enumerate}
\item getopt
\begin{lstlisting}[language=c, numbers=none]
#include <getopt.h>

int getopt(int argc, char *argv[], const char *optstring, char **parg);

description:
	split argv by optstring and return by parg. every call, return a value.
argument:
	argc--the numbers of arguments
	argv--arguments list
	optstring--options string
	parg----a string which return after match one argument.
\end{lstlisting}
\end{enumerate}
