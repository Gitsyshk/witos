\section{存储设备操作}

\subsection{flash}
\textbf{usage}: flash <子命令> [可选参数 <值>]

\subsubsection{子命令:}
\begin{table}[H]
%\centering
\setlength{\parindent}{0pt}
\begin{tabular}{|c|l|} \hline
子命令名称 & 命令说明 \\ \hline
info & 显示相应分区或者设备的信息 \\ \hline
dump & 查看flash页内容,以page为单位,包括oob \\ \hline
read & 从Flash中加载数据到DDR \\ \hline
write & Flash中写数据from DDR \\ \hline
erase & 以block为单位,擦除flash一块内容 \\ \hline
scanbb & 扫描指定分区上的坏快 \\ \hline
\end{tabular}
\end{table}

%fixme: specific the size to read or write.
\subsubsection{命令选项:}
\begin{enumerate}
	\item 通用选项:
	\begin{itemize}
		\item -a <address> \\
		该地址的单位可以为:byte,page,block,K,M,G bytes。
	\end{itemize}

	\item 子命令选项:
	\begin{enumerate}
		\item dump
		\begin{itemize}
			\item -d <d|D|o|O|x|X> \\
			分别以十进制、八进制和十六进制输出,其中选项大写表示。 %fixme
		\end{itemize}

		\item read/write
		\begin{itemize}
			\item -f <file> \\
			其中file指需要读写的文件。
			\item -m <memory> \\
			memory地址按字节表示,它可能会影响到image的跳转地址。
		\end{itemize}

		\item erase
		\begin{itemize}
			\item -c <size> \\
			cleanmark size for JFFS2. %fixme
		\end{itemize}
	\end{enumerate}
\end{enumerate}

\subsubsection{命令使用示例:}
\begin{lstlisting}[numbers=none]
$ flash dump -a 1M -d x
$ flash read -a 100block -f file
\end{lstlisting}

\subsection{disk}
\textbf{usage}: disk <子命令> [可选参数 <值>] \\
\\ 通用的磁盘驱动工具,支持MMC/eMMC/SD,ATA,SCSI等。

\subsubsection{子命令:}
\begin{table}[H]
\setlength{\parindent}{0pt}
\begin{tabular}{|c|l|} \hline
子命令名称 & 命令说明 \\ \hline
scan & 遍历并显示所有分区或者设备的信息 \\ \hline
dump & 以块为单位读并且显示所有设备数据 \\ \hline
write & 将数据由DDR写到磁盘中 \\ \hline
\end{tabular}
\end{table}

\subsubsection{命令选项:}
\begin{enumerate}
	\item -a <address>
	指定磁盘的起始地址,以字节或者扇区为单位。
	\item -m <memory>
	指定内存起始地址。
	\item -w <1|2|4|8>
	指定每一个数据单元大小,其单位为字节。
\end{enumerate}

\subsection{ls}
\textbf{usage}: ls [<可选选项>] \\
\\ 显示当前文件夹或者块设备的目录信息。

\subsubsection{命令选项:}
\begin{itemize}
	\item -l \\
	显示详细信息,该参数为默认参数。
\end{itemize}

\subsection{cd}
\textbf{usage}: cd [<vol>] \\
\\ 切换分区,默认为home。

\section{网络操作}

\subsection{ipconfig}
\textbf{usage}: ipconfig [<NIC>] [<IP>] [-n <netmask>] [-m <MAC>] [-s] [-h] \\
\\ 网络接口参数配置工具。

\subsubsection{命令选项:}
\begin{itemize}
	\item <NIC> \\
	指定网络设备对象,如``eth0''。缺省为系统中第一个网络设备。
	\item <IP> \\
	配置网络设备IP地址。
	\item -n <netmask> \\
	指定子网掩码。
	\item -m <MAC> \\
	指定网卡MAC地址。
	\item -s \\
	将配置信息保存到系统配置文件。
\end{itemize}

\subsubsection{不加任何option时显示interface的信息,具体包括:}
\begin{enumerate} \setlength{\itemsep}{-\itemsep}
	\item NIC芯片名称(ID及字符串表示)
	\item PHY信息:ID、地址
	\item 连接状态,包括速率
	\item RX/TX bytes
	\item error count
\end{enumerate}

\subsection{ping}
\textbf{usage}:ping [<可选选项>]

\subsubsection{命令选项:}
\begin{itemize}
	\item -c <count> \\
	指定发送ICMP包的数目。
\end{itemize}

\subsection{tftp}
\textbf{usage}:tftp [可选子命令 <参数值>] \\
\\ 简单文件传输协议工具。

\subsubsection{子命令:}
\begin{table}[H]
\setlength{\parindent}{0pt}
\begin{tabular}{|c|l|} \hline
子命令名称 & 命令说明 \\ \hline
get & 下载文件 \\ \hline
put & 上传文件 \\ \hline
\end{tabular}
\end{table}

\subsubsection{命令选项:}
\begin{enumerate}
	\item 通用选项:
	\begin{itemize}
		\item -r <URL> \\
		指定远端URL地址,如``10.0.0.2:69/g-bios.bin''。
		\item -l <path> \\
		指定本地路径。%fixme
		\item -m <mode> \\
		指定tftp传输模式:文本或者二进制(默认为二进制)。
		\item -t <type> \\
		指定烧录image文件的类型,默认为自动检测。
		\item -v \\
		详细显示操作信息。
	\end{itemize}

	\item 子命令专有选项:
	\begin{itemize}
		\item -a <address> \\
		下载到指定内存,不写回存储介质(该命令可能会影响到go命令)。
	\end{itemize}
\end{enumerate}

\subsection{ftp}
\textbf{usage}: ftp [<可选子命令>] [<可选参数>] [[user[:password]]@URL] \\
\\ ftp 传输工具。

\subsubsection{子命令:}
\begin{table}[H]
\setlength{\parindent}{0pt}
\begin{tabular}{|c|l|} \hline
子命令名称 & 命令说明 \\ \hline
get & 从ftp服务器上下载文件(默认情况下为下载)\\ \hline
put & 上传指定文件到服务器上 \\ \hline
\end{tabular}
\end{table}

\subsubsection{命令选项:}
\begin{enumerate}
	\item 通用选项:
	\begin{itemize}
		\item -v \\
		详细显示文件传输过程。
	\end{itemize}

	\item 专有选项:
		\begin{enumerate}
			\item get
			\begin{itemize}
				\item -m <memory> \\
				仅仅下载到内存,它可能改变系统的跳转地址。 %fixme
			\end{itemize}

			\item put
			\begin{itemize}
				\item -f <file> \\
				指定上传的文件。
			\end{itemize}
		\end{enumerate}
\end{enumerate}

\subsection{dhclient}
\textbf{usage}:dhclient [<可选选项>] \\
\\ DHCP客户端工具。

\subsubsection{命令选项:}
\begin{itemize}
	\item -s \\
	更新系统配置文件中DHCP服务器地址。
	\item -x <NIC> \\
	指定网络接口。
\end{itemize}

\section{串口协议及工具}
\subsection{uart}
\textbf{usage}: uart <子命令> [<可选参数值>] \\
\\ uart串口传输工具。

\subsubsection{子命令:}
\begin{table}[H]
\setlength{\parindent}{0pt}
\begin{tabular}{|c|l|} \hline
子命令名称 & 命令说明 \\ \hline
load & 下载文件\\ \hline
send & 上传文件 \\ \hline
setup & 设置uart串口传输接口\\ \hline
test & 向主机发送测试字符(检测串口) \\ \hline
\end{tabular}
\end{table}

\subsubsection{命令选项:}
\begin{enumerate}
	\item 通用选项:
	\begin{itemize}
		\item -p <k|kermit|y|ymodem> \\
		选择串口传输协议(kermit 或者 ymodem),默认值由配置文件给出。
		\item -f <file>
		指定传输的文件。
		\item -i <name>|<num> \\
		指定UART接口ID,如数字0、1、2等,或者字符串``ttyS0''、``ttySAC1''等,默认值由配置文件给出。
		\item -m [<address>] \\
		将下载的数据放在内存中,而不直接写到storage(如Flash)上;若不指定address,则由系统自动分配一块空间。(它可能会影响到go命令)
		\item -v \\
		详细显示每步操作信息。
	\end{itemize}
\end{enumerate}

\section{内存读写及指令跳转}

\subsection{mem}
\textbf{usage}: mem <子命令> [<可选参数>] \\
\\ 内存调试工具

\subsubsection{子命令:}
\begin{table}[H]
%\centering
\begin{tabular}{|c|l|} \hline
命令名称 & 命令说明 \\ \hline
move & memory move \\ \hline
copy & memory copy \\ \hline
write & memory write \\ \hline
set & 将某个memory空间写入值 \\ \hline
dump & 以十六进制或者字符型显示内存值 \\ \hline
free & 显示内存 free/used 信息。\\ \hline
\end{tabular}
%\caption{md}
\end{table}

\subsubsection{命令选项:}
\begin{enumerate}
	\item move/copy
	\begin{itemize}
		\item -s <src> \\
		指定源数据地址。
		\item -d <dst> \\
		指定目标地址。
		\item -l <size> \\
		移动或者复制数据大小,单位可以为K、M或者G字节,默认为字节。
	\end{itemize}

	\item set \\
	\textbf{usage}: mem set [<选项>] [<参数值>, ...]
	\begin{itemize}
		\item -a <address> \\
		指定写入内存的起始地址。
		\item -w <1|2|4|8> \\
		基本操作单元,默认为1字节。
	\end{itemize}

	\item dump
	\begin{itemize}
		\item -a <address> \\
		指定内存的起始地址。
		\item -l <len> \\
		指定显示数据的大小。
		\item -d <o|O|d|D|x|X> \\
		指定显示的格式,默认为十六进制。
	\end{itemize}
\end{enumerate}

\subsection{go}
\textbf{usage}: go [<address>] \\
\\ 跳转的指定目标地址执行,可十进制表示,也可十六进制表示。

\section{系统工具}
\subsection{sysconf}
\textbf{usage}: config [<可选选项>] [<可选属性> [<值>]] \\
\\ g-bios 系统配置工具。

\subsubsection{命令选项:}
\begin{itemize}
	\item -a <属性> <属性值> \\
	增加一个新的属性,并赋值。
	\item -d <属性> \\
	删除属性。
	\item -s <属性> <属性值> \\
	给某一属性重新赋值。
	\item -g <属性> \\
	获取某一属性值。
	\item -l \\
	显示所有属性信息,默认设置。
\end{itemize}

\subsection{boot}
\textbf{usage}: boot [OS] [<选项>] \\
\\ g-bios 启动 kerneli 配置工具。

\subsubsection{命令选项:}
\begin{itemize}
	\item -t [kernel] \\
	通过tftp下载kernel image。
	\item -s <mmc:kernel> \\
	通过mmc方式加载kernel。
	\item -f [<part\_num>] \\
	指定rootfs目录,如:root=/dev/mtdblock<part\_num>
	\item -n [nfs\_server:] \\
	通过nfs 方式加载rootfs。
	\item -l [kernel command line] \\
	指定kernel启动的commandline。
	\item -r [ramdisk\_image\_name] \\
	通过ramdisk启动kernel。
	\item -c <console\_name, baudrate> \\
	设置控制台名称以及波特率。%fixme
	\item -m <machine ID> \\
	指定板子的machine ID。
	\item -v \\
	打印linux启动参数并退出。
\end{itemize}

\subsection{reboot}
\textbf{usage}: reboot [<可选选项>] \\
\\ 重启系统。

\subsubsection{命令选项:}
\begin{itemize}
	\item -v \\
	再重启时显示更详细的信息。
\end{itemize}


% \section{flash读写及分区}
% \subsection{flash}
% usage: flash <子命令> [可选参数 <值>]
%
% \begin{table}[H]
% %\centering
% \setlength{\parindent}{0pt}
% \begin{tabular}{|c|l|} \hline
% 子命令名称 & 命令说明 \\ \hline
% %select & 选中某个flash设备,作为其他flash命令的操作对象 \\ \hline
% dump & 查看flash页内容,以page为单位,包括oob \\ \hline
% read & 从Flash中加载数据到DDR \\ \hline
% write & Flash中写数据from DDR \\ \hline
% erase & 以block为单位,擦除flash一块内容 \\ \hline
% scanbb & 扫描指定分区上的坏快 \\ \hline
% \end{tabular}
% %\caption{Flash子命令}
% \end{table}
%
% \begin{table}[H]
% \setlength{\parindent}{0pt}
% \begin{tabular}{|c|l|} \hline
% 参数 & 功能 \\ \hline
% -a & 设定起始地址\\ \hline
% -l & 设定长度\\ \hline
% -p & 设定分区,不与-a和-l并用\\ \hline
% -m & mem的起始址址\\ \hline
% -f & 强制erase(无论是否存在坏块)\\ \hline
% -c & 擦除的同时写入cleanmark\\ \hline
% \end{tabular}
% \end{table}
%
% 命令使用示例:\\
% \begin{itemize}
% \item flash erase -a 1M -l 32K
% \item flash erase -a 100block -l 16block -f
% \end{itemize}
%
% \subsection{part}
% usage: part [options]
%
% 参数说明:
%
% \begin{table}[H]
% \begin{tabular}{|c|l|}
% \hline
% 参数 & 功能 \\ \hline
% -l & 显示分区表 \\ \hline
% %-i [N] & 显示分区N的具体信息 \\ \hline
% \end{tabular}
% \end{table}
%
% \subsection{ls}
% 显示当前分区的具体信息
%
% \subsection{cd}
% 切换分区
%
% \section{MMC/SD卡操作}
% \subsection{mmc}
% usage: mmc $<$子命令$>$ [options]
%
% \begin{table}[H]
% %\centering
% \setlength{\parindent}{0pt}
% \begin{tabular}{|c|l|} \hline
% 子命令名称 & 命令说明 \\ \hline
% scan & 扫描所有MMC/SD设备,解析并打印设备信息 \\ \hline
% %select & 选中某个MMC/SD设备,作为其他MMC/SD命令的操作对象 \\ \hline
% dump & 查看MMC/SD上的数据,每次显示一个block \\ \hline
% %read & 从MMC/SD中加载数据到DDR \\ \hline
% %write & MMC/SD中写数据from DDR \\ \hline
% %erase & 以block为单位,擦除MMC/SD一块内容 \\ \hline
% \end{tabular}
% %\caption{MMC子命令}
% \end{table}
%
% MMC/SD命令选项及参数介绍:
% \begin{table}[H]
% \setlength{\parindent}{0pt}
% \begin{tabular}{|c|l|} \hline
% 参数 & 功能 \\ \hline
% -a & 设定起始地址\\ \hline
% %-l & 设定长度\\ \hline
% %-m & mem的起始址址\\ \hline
% \end{tabular}
% \end{table}
%
% \section{网络连接}
%
% \subsection{ifconfig}
% usage: ifconfig [interface] [address] [netmask <address>] [hw [HW] <address>]
%
% 选项及参数介绍:
% \begin{table}[H]
% \setlength{\parindent}{0pt}
% \begin{tabular}{|l|l|} \hline
% 选项 & 功能描述 \\ \hline
% interface & 指定网络设备对象,如``eth0''。缺省为系统中第一个网络设备。 \\ \hline
% address & 配置网络设备IP地址为address \\ \hline
% netmask <address> & 配置netmask为address \\ \hline
% hw [HW] <address> & 配置设备的MAC地址为address。其中HW缺省为``ether'' \\ \hline
% \end{tabular}
% \end{table}
%
% 不加任何option时显示interface的信息,具体包括:
% \begin{enumerate} \setlength{\itemsep}{-\itemsep}
% \item NIC芯片名称(ID及字符串表示)
% \item PHY信息:ID、地址
% \item 连接状态,包括速率
% \item RX/TX bytes
% \item error count
% \end{enumerate}
%
% \subsection{ping}
% usage:ping [DestIp]
%
% 若DestIp不指定,则默认的server ip
%
% \subsection{tftp}
% usage:tftp [options] [filename]
%
% 参数介绍:
% \begin{table}[H]
% \setlength{\parindent}{0pt}
% \begin{tabular}{|c|l|} \hline
% 选项 & 功能描述 \\ \hline
%  -s & 设定服务端IP \\ \hline
%  -m & 下载的内容放在内存里,即,不烧录到flash上 \\ \hline
% \end{tabular}
% \end{table}
%
% \subsection{dhclient}
% usage:dhclient [options]
%
% 参数介绍:
% \begin{table}[H]
% \setlength{\parindent}{0pt}
% \begin{tabular}{|c|l|} \hline
% 选项 & 功能描述 \\ \hline
%  -s & 同时将Server IP更新为DHCP Server \\ \hline
% \end{tabular}
% \end{table}
%
% \section{串口协议及工具}
%
% \subsection{kermit}
% usage:kermit [options]
%
% 作用概述:串口文件传输%\footnote{目前kermit和ymodem仅支持下载}
%
% 选项及参数介绍:
% \begin{table}[H]
% \setlength{\parindent}{0pt}
% \begin{tabular}{|c|m{10cm}|} \hline
% 选项 & 功能描述 \\ \hline
%  -m [address] & 将下载的数据放在内存中,而不直写到storage(如Flash)上。其中address为可选参数,表示memory地址;若不指定address,则由系统自动分配一块空间。 \\ \hline
% \end{tabular}
% \end{table}
%
% \subsection{ymodem}
% usage: ymodem [options]
%
% 作用概述:串口文件传输%\footnote{目前kermit和ymodem仅支持下载}
%
% 选项及参数介绍:
% \begin{table}[H]
% \setlength{\parindent}{0pt}
% \begin{tabular}{|c|m{10cm}|} \hline
% 选项 & 功能描述 \\ \hline
%  -m [address] & 将下载的数据放在内存中,而不直写到storage(如Flash)上。其中address为可选参数,表示memory地址;若不指定address,则由系统自动分配一块空间。 \\ \hline
% \end{tabular}
% \end{table}
%
% \section{Graphics和Display}
% \subsection{lcd}
% usage:lcd [options]
%
% 参数介绍:
% \begin{table}[H]
% \setlength{\parindent}{0pt}
% \begin{tabular}{|c|l|} \hline
% 选项 & 功能描述 \\ \hline
% -l [all] & 列出LCD的video mode。all表示所有video mode,不加则仅显示当前的video mode \\ \hline
% -s <N> & 将当前LCD的video mode设置为第N种mode。 \\ \hline
% \end{tabular}
% \end{table}
%
% \section{memory读写及指令跳转}
%
% \subsection{mem}
% \begin{table}[H]
% %\centering
% \begin{tabular}{|c|l|}
% \hline
% 命令名称 & 命令说明\\ \hline
% read & 显示memory数据\\ \hline
% write & 将数据写入memory\\ \hline
% set & 将某个memory空间写入值\\ \hline
% \end{tabular}
% %\caption{md}
% \end{table}
%
% \subsection{go}
% usage: go <address>
%
% address跳转的目标地址,可十进制表示,也可十六进制表示。
%
% 示例: go 0xc000000 跳转到0xc000000处执行。
%
% \section{系统配置}
% \subsection{sysconf}
% \begin{table}[H]
% %\centering
% \begin{tabular}{|c|l|}
% \hline
% 命令名称 & 命令说明\\ \hline
% -r <all|net|boot> & sysconf reset \\ \hline
% \end{tabular}
% %\caption{md}
% \end{table}
%
% \section{其他命令}
% \subsection{help}
% 列出g-bios系统中当前所有可用的命令
%
% \subsection{led}
% LED灯测试
