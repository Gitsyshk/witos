\section{S3C24XX系列平台}

\section{AT91SAM系列平台}

\section{S3C64XX系列平台}
S3C64XX and S5P as cases

\section{OMAP3系列平台}
\subsection{devkit8000开发板}
将g-bios烧录到SD卡的步骤如下:
\begin{enumerate}
\item make devkit8000\_defconfig
\item make
\item make install将witrom.bin以及g-bios.bin拷贝到默认目录/var/lib/tftpboot下
\item 将SD卡插入PC,创建一个fat32分区,并将其设置为启动分区,具体步骤如下:
 \begin{enumerate}
  \item 依据读卡器的不同,sd卡插入后会在dev目录下产生sdb或者mmcblk0文件,执行如下命令sudo fdisk /dev/sdb或者/dev/mmcblk0
  \item 依照提示,按下n(add a new partition)
  \item 按下p键(创建primary partition)
  \item 按下1
  \item 一路回车
  \item 按下a键(toggle a bootable flag)
  \item 按下1
  \item 按下t键(change a partition's system id)
  \item 按下l键
  \item 选择c(W95 FAT32(LBA))
  \item 按下w键
  \item 按下q键退出fdisk
  \item 执行如下命令:sudo mkfs.vfat /dev/sdb1或者/dev/mmcblk0p1,将分区格式成FAT32
 \end{enumerate}
\item 将sd卡上的FAT32分区mount到pc上后,将witrom.bin拷贝到sd的FAT32分区上,并将其改名为MLO文件
\item umount sd卡
\end{enumerate}
从SD卡启动方法如下:
将SD插入开发板,按住boot键,开启电源(此时是从SD卡启动,若直接开启电源,则从NAND启动)
