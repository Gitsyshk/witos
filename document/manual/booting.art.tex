
%\section{Add New Commands}
%g-bios添加命令规范。
%\begin{lstlisting}
%static char app_option[][CMD_OPTION_LEN] = {};
%
%#include <getopt.h>
%
%int getopt(int argc, char *argv[], const char *optstring, char **arg);
%\end{lstlisting}

boot command design ..

\section{OS引导}
见第五章?
\begin{table}[H]
\setlength{\parindent}{0pt}
\begin{tabular}{|c|c|}
\hline
 命令名称 & 命令说明\\
\hline
boot & 引导操作系统\\
\hline
\end{tabular}
\end{table}

\noindent{}命令名称:boot\\
参数介绍:\\
\begin{table}[H]
\setlength{\parindent}{0pt}
\begin{tabular}{|p{2.5cm}|p{8.5cm}|}
\hline
-t [filename] &若指定filename,则通过tftp下载kernel image文件;否则从本地的linux分区下载kernel image文件 \\ \hline
-r [filename] &用ramdisk启动。指定filename,则通过tftp下载ramdisk image;否则从本地ramdisk分区下载。 \\ \hline
-f [N] & 指定rootfs分区,N为分区号 \\ \hline
-n [ip:path] & 用nfs方式mount rootfs \\ \hline
-v &仅显示kernel启动参数,但并不真正引导OS \\ \hline
\end{tabular}
\end{table}
\section{TFTP + NFS}

\indent 其中NFS服务配置和编译linux kernel部分详情请参阅$<<$MaxWit Lablin开发者手册$>>$第一卷\\
\indent 在g-bios命令行下,输入:\\

\begin{verbatim}
g-bios: 0# boot -t zImage -n 192.168.0.2:/home/maxwit/maxwit/rootfs
【说明】
-t [filename]:用tftp方式下载指定的kernel image
-n [nfs_server:/nfs/path/]: 用NFS方式mout rootfs。也可以加上参数,如:-n 192.168.0.111:/path/to/nfs
\end{verbatim}

boot程序具有记录功能,即,能记住用户输入的参数,换句话,再次输入boot时不再需要输入参数了,除非你想重设参数。

\section{FLASH + NFS}
\begin{verbatim}
g-bios: 1# cd 3  (进入Linux分区)
g-bios: 3# ls (列出当前分区信息)
        Partition Type = "linux"
        Partition Base = 0x00080000 (512K)
        Partition Size = 0x00200000 (2M)
        Host Device    = NAND 256MB 3.3V 8-bit
        MTD Deivce     = /dev/mtdblock3
        Image File     = "zImage" (1968220 bytes)
g-bios: 3# tftp zImage (下载zImage 到当前分区)
 "zImage": 192.168.2.101 => 192.168.2.100
 1968220(1M898K92B) loaded

g-bios: 1# boot  -t  -n 192.168.2.11:/root/maxwit/rootfs
【说明】
-t  不加参数,从Linux分区Load kernel image
-n [nfs_server:/nfs/path/]: 用NFS方式mount rootfs。也可以加上参数。如:-n 192.168.0.111:/home/maxwit/maxwit/rootfs。
\end{verbatim}

\section{Booting from Flash}

\begin{verbatim}
g-bios: 1# cd 3  (进入Linux分区)
g-bios: 3# ls (列出当前分区信息)
        Partition Type = "linux"
        Partition Base = 0x00080000 (512K)
        Partition Size = 0x00200000 (2M)
        Host Device    = NAND 256MB 3.3V 8-bit
        MTD Deivce     = /dev/mtdblock3
        Image File     = "zImage" (1968220 bytes)
g-bios: 3# tftp zImage (下载zImage 到当前分区)
 "zImage": 192.168.2.101 => 192.168.2.100
 1968220(1M898K92B) loaded

g-bios: 3# cd 5 (进入Rootfs分区)
g-bios: 5# tftp rootfs_l.jffs2 (下载zImage 到当前分区)
g-bios: 5# boot -t -f 5
【说明】
-t :不加参数,从Linux分区Load kernel image
-f [N]:指定rootfs的分区,N为分区号
\end{verbatim}
